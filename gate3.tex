\documentclass[journal,12pt,twocolumn]{IEEEtran}
\usepackage{amsmath,amssymb,amsfonts,amsthm}
\usepackage{txfonts}
\usepackage{tkz-euclide}
\usepackage{listings}
\usepackage{gvv}
\usepackage[latin1]{inputenc}
\usepackage{array}
\usepackage{pgf}
\usepackage{lmodern}
\usepackage{amsmath}
\usepackage{circuitikz}
\begin{document}
\bibliographystyle{IEEEtran}

\title{GATE 2022[IN]-64}
\author{EE23BTECH11066 - Yakkala Amarnath Karthik}
\maketitle
\bibliographystyle{IEEEtran}

\textbf{Question:}\\ \\
In the circuit shown, the switch is initially closed. It is opened at t= 0 s and
remains open thereafter. The time (in milliseconds) at which the output voltage
$V_{out}$ becomes LOW is  (round off to three decimal places) \hfill(GATE 2022 IN)\
\input{figs/tikzque}

\textbf{Solution:}\\ \\
At t$=0^-$, when the switch is closed,\\
The voltage across the capacitor is:
\begin{align}
V_c\brak{0^-}&=5\times\frac{5}{5+1}\\
&=\frac{25}{6}V
\end{align}
$V_c\brak{0^-}$ is also the non inverting voltage of the OP-AMP\\ \\
At $t=0^+$, when the switch is open,\\
The voltage across inverting terminal is:
\begin{align}
V_I&=5\times\frac{600}{600+400}\\
&=2V
\end{align}
Immediately after the switch is open, voltage across capacitor do not change.\\ So $V_{NI}>V_I$,Hence the output of OP-AMP is fixed to +5V.
Later, Capacitor discharges into 5K$\Omega$ resistor.\\
The discharging equation is as follows:
\begin{align}
    V_C\brak{t}&=V_C\brak{0^-}e^{\frac{-t}{\tau}}\\ 
    2&=\frac{25}{6}\times e^{\frac{t_0}{RC}}\\
    t&=RC\ln\brak{\frac{25}{12}}\\
    &=0.1\times10^{-6}\times5\times10^{3}\ln{\brak{\frac{25}{12}}}\\
    t&=0.367ms
\end{align}

\input{figs/tikzans}

\end{document}
